\documentclass[12pt]{article}
\usepackage[utf8]{inputenc}
\usepackage{geometry}
\usepackage{graphicx}
\usepackage{booktabs}
\usepackage{amsmath}
\usepackage{hyperref}
\usepackage{float}
\usepackage{titlesec}

% Set margins
\geometry{a4paper, margin=1in}

% No indentation for paragraphs
\setlength{\parindent}{0pt}
\setlength{\parskip}{1em}

\title{\textbf{Tax-Loss Harvesting: Alpha, Cash Drag, and Optimization}}
\author{Anonymous Team}
\date{December 2025}

\begin{document}

\maketitle

\begin{abstract}
This paper investigates the efficacy of Tax-Loss Harvesting (TLH) strategies applied to a Direct Indexing portfolio tracking the S\&P 500 over the period 2014-2024. We compare active harvesting strategies against a passive benchmark under specific wash-sale constraints. Our results demonstrate that while harvesting generates theoretical alpha, the ``cash drag'' mandated by the IRS Wash Sale Rule significantly erodes these gains in a strong bull market. We further analyze the impact of different cash-flow scenarios, finding that a charitable giving liquidation strategy maximizes final wealth retention.
\end{abstract}

\section{Introduction}

Tax-loss harvesting is a widely utilized active management strategy aimed at enhancing after-tax returns by realizing capital losses to offset gains. The central premise is that the tax savings generated from realized losses can be reinvested, creating a compounding ``tax alpha.'' However, regulatory constraints---specifically the IRS Wash Sale Rule---prohibit the immediate repurchase of ``substantially identical'' securities within 30 days of a loss realization. This creates a mandatory lockout period, introducing tracking error and opportunity cost, commonly referred to as ``cash drag.''

In this study, we simulate the performance of a tax-aware Direct Indexing portfolio constructed from the top 50 constituents of the S\&P 500. We aim to quantify the trade-off between the tax benefits of harvesting and the performance drag caused by wash-sale compliance. We evaluate this trade-off across two distinct investor lifecycles: a standard income-withdrawal model and a tax-efficient charitable giving model.

\section{Data and Benchmark Construction}

Our universe consists of the top 50 constituents of the S\&P 500, selected to ensure high liquidity and data availability. Daily Adjusted Close prices were sourced via \texttt{yfinance} for the period January 1, 2014, to January 1, 2024.

\subsection{Benchmark Portfolio}
The benchmark is a passive, equal-weighted portfolio of the universe, rebalanced monthly. It incurs no transaction costs and performs no tax harvesting. This serves as the control group ($R_B$) against which the active strategies ($R_A$) are measured.

\subsection{Tax Modeling Assumptions}
To isolate the economic value of the tax benefit, we model the ``Tax Alpha'' as an immediate capital injection.
\begin{itemize}
    \item \textbf{Tax Rate ($\tau$):} A flat rate of 20\% is applied to all realized gains and losses.
    \item \textbf{Tax Credit:} Upon realizing a loss of $L$, the portfolio is immediately credited with cash equal to $L \times \tau$. This simulates an investor who uses these losses to offset external gains elsewhere, effectively freeing up capital for immediate reinvestment.
\end{itemize}

\section{Methodology and Strategies}

We evaluate three distinct active strategies to test the sensitivity of returns to wash-sale constraints.

\subsection{Strategy 1: Greedy (No Wash Rule)}
This strategy scans the portfolio daily. If any position's price drops more than 5\% below its cost basis, it is liquidated to realize the loss. Crucially, the strategy \textbf{immediately repurchases} the same security.
\begin{itemize}
    \item \textbf{Hypothesis:} This represents the theoretical upper bound of Tax Alpha ($TA_{ideal}$), as it maintains perfect market exposure while harvesting all available losses.
    \item \textbf{Constraint:} This violates the IRS Wash Sale Rule and is for theoretical comparison only.
\end{itemize}

\subsection{Strategy 2: Greedy (With Wash Rule)}
This strategy strictly adheres to IRS regulations. When a loss is harvested:
1. The security is sold.
2. The ticker is added to a ``Restricted List'' for 30 days.
3. The proceeds are held in \textbf{Cash} (yielding 0\%) until the restriction expires.
\begin{itemize}
    \item \textbf{Hypothesis:} This strategy will suffer from ``Cash Drag'' ($CD$), defined as the return difference between the market and cash during the lockout period.
\end{itemize}

\subsection{Strategy 3: Optimized Tax-Aware}
We implement a convex optimization framework (\texttt{cvxpy}) to minimize tracking error against the benchmark.
\begin{itemize}
    \item \textbf{Objective:} $\text{Minimize } (w - w_B)^T \Sigma (w - w_B)$
    \item \textbf{Constraint:} $w_i = 0$ if $i \in \text{Restricted List}$.
    \item \textbf{Mechanism:} Instead of holding cash, the optimizer attempts to reallocate capital to other correlated stocks in the universe to maintain market exposure.
\end{itemize}

\section{Empirical Results (2014-2024)}

The simulation covered a decade characterized by a strong equity bull market. The results for the \textbf{Income Withdrawal Scenario} (5\% annual withdrawal) are summarized below:

\begin{table}[H]
\centering
\begin{tabular}{lrrr}
\toprule
\textbf{Strategy} & \textbf{Final Wealth} & \textbf{Total Tax Paid} & \textbf{Realized Losses} \\
\midrule
Baseline (Passive) & \$29,779,543 & \$2,789,422 & \$1,769,279 \\
\textbf{Greedy (No Wash)} & \textbf{\$32,570,308} & \$2,675,905 & \$6,439,701 \\
Greedy (With Wash) & \$25,184,912 & \$4,417,664 & \$22,607,123 \\
Optimized & \$25,700,917 & \$4,378,182 & \$22,543,163 \\
\bottomrule
\end{tabular}
\caption{Comparison of Final Wealth and Tax Metrics (Income Scenario)}
\label{tab:results}
\end{table}

\subsection{The ``Cash Drag'' Penalty}
The most striking finding is the underperformance of the compliant strategies. The \textbf{Greedy (With Wash)} strategy underperformed the baseline by over \$4.5M.
\begin{itemize}
    \item \textbf{Analysis:} In a rising market, the opportunity cost of being out of the market (Cash Drag) exceeds the 20\% value of the harvested tax credit. The ``No Wash'' strategy's outperformance confirms that the \textit{harvesting} itself is valuable, but the \textit{regulatory friction} destroys that value in this specific universe and market regime.
\end{itemize}

\subsection{Charitable Giving Scenario}
We also tested a ``Charitable Giving'' scenario where the investor contributes \$1M annually and donates the portfolio at Year 10 (avoiding terminal liquidation tax).
\begin{itemize}
    \item \textbf{Baseline Final Wealth:} \$75,963,882
    \item \textbf{No Wash Final Wealth:} \$83,256,160
    \item \textbf{Conclusion:} The combination of tax-free donation and tax-loss harvesting (in the ideal case) creates significant wealth compounding.
\end{itemize}

\section{Visualizations}

\begin{figure}[H]
    \centering
    \includegraphics[width=0.9\textwidth]{wealth_curves_wealth_over_time_-_income_withdrawal.png}
    \caption{Wealth Accumulation over 10 Years (Income Scenario)}
    \label{fig:wealth}
\end{figure}

\begin{figure}[H]
    \centering
    \includegraphics[width=0.9\textwidth]{drawdowns_drawdowns_-_income_withdrawal.png}
    \caption{Drawdown Analysis}
    \label{fig:drawdowns}
\end{figure}

\section{Conclusion}

Our analysis suggests that for a Direct Indexing portfolio of 50 stocks, standard Tax-Loss Harvesting strategies may be detrimental during strong bull markets due to wash-sale constraints. The ``Cash Drag'' of waiting 30 days or the tracking error of using imperfect substitutes can outweigh the tax benefits.

Future work should explore expanding the universe to 500+ stocks or using ETFs as substitutes to minimize the time spent in uninvested cash, thereby capturing the tax alpha demonstrated by the ``No Wash'' theoretical model.

\section{References}
\begin{enumerate}
    \item Chaudhuri, S., Burnham, T., \& Lo, A. (2020). ``An Empirical Evaluation of Tax-Loss Harvesting Alpha.''
    \item Moehle, N., et al. (2021). ``Tax-Aware Portfolio Construction via Convex Optimization.''
\end{enumerate}

\end{document}

